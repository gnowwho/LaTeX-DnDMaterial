\section{Olorin}

\subsection{Nascita e perdita della madre}

Olorin è il figlio bastardo di un Visconte del Territorio di Enador: Karl Puffenstein (\textit{da cambiare}).  %Olortir.
Nel territorio della Viscontea, nonostante il clima di insofferenza nei confronti dei non umani diffuso nel resto del regno, risiede da tempo una piccola comunità elfica, piuttosto isolata dalla popolazione, ma tollerata per il contributo che il druido che la guida garantisce agli agricoltori locali.

La madre, un'elfa di nome Gnegnarella (\textit{da cambiare}), di stirpe draconica, è una delle numerose esuli della controffensiva Yuan-ti contro il Reame Silvano nell'est. Portata via dalle macerie della città sospesa di Luhr-yen dalla madre ancora in giovanissima età, affrontò il viaggio verso ovest, fuggendo dalla guerra e dalle aspettative del padre, anch'egli erede del drago.

Raggiunto il regno di Oster e congiuntosi con uno dei relativamente scarsi enclavi elfici sul territorio, nella viscontea dei Puffenstein, seppur osteggiata per la sua relativa estraneità al gruppo, e la sua differenza raziale (in quanto Alta Elfa, tra Elfi dei Boschi) venne comunque accolta e protetta all'interno della comunità anche per pietà nei confronti della figlia di pochi anni, praticamente neonata per gli standard elfici.

Gnegnarella crebbe quindi all'interno della comunità, senza conoscere la verità sulla propria linea di sangue, pur dimostrando le capacità innate che essa le forniva. La madre, con anche l'appoggio del druido che guidava la comunità, tentò di indirizzarla alle arti druidiche per allontanarla il più possibile da quel potere, in lei innato, che temeva e odiava.

Proprio per via del suo percorso di apprendimento druidico Gnegnarella ebbe diverse occasioni di lasciare il bosco in cui la comunità si riparava ed entrare in contatto con la popolazione umana circostante. Questo, insieme alla sua relativa marginalizzazione nei confronti della comunità in cui viveva, la spinse ad adottare alcuni atteggiamenti meno propri della comunità elfica e in fine, a conoscere l'ancora giovane Karl.

I rapporti tra i due si spinsero all'intimità, e Gnegnarella rimase incinta di Olorin all'età, tenera per gli standard elfici, di soli 47 anni. Un avvenimento simile attirò su di lei ulteriore astio e disappunto da parte della comunità in cui viveva, e attirò persino il biasimo della madre che, tuttavia la supportò nella gravidanza.

In parte per le pressioni della madre, ed in parte per il suo desiderio di dare una vita al figlio che non lo esponesse all'astio della propria cominità, grazie anche alla giovanile ingenuità di Karl, Olorin venne consegnato a quest'ultimo subito dopo lo svezzamento e riconosciuto, prima che il padre di Karl potesse intervenire per evitare la cosa.

Negli anni Olorin avrà pochi contatti con la madre, che vedrà comunque come una vittima di una situazione causata dal padre, notevolmente più freddo e calcolatore nel momento in cui Olorin avrà l'età per giudicarlo. Ella si ritirerà all'interno della comunità in sempre maggiore isolamento, dedicandosi asceticamente alle pratiche druidiche.
Fu però dopo la morte di Re Gustav II, con l'aumento di influenza della chiesa e delle persecuzioni che ciò che aveva protetto la comunità elfica fino ad allora divenne la sua condanna: la presenza di druidi venne infatta vista come eresia e condannata. Un corpo di Assolutori guidati da un inquisitore attaccò e bruciò il bosco in cui la comunità si rifugiava, distruggendone una parte e uccidendo molti degli abitanti. Il Druido che guidava la comunità fu giustiziato dopo un processo per eresia, e molti dei morti non vennero mai riconosciuti a causa delle fiamme.

Olorin reagì molto male alla totale assenza di interventi da parte del padre, seppur conscio delle poche possibilità di intervento di quest'ultimo, rinconoscendo in lui, dopo che nella chiesa, una delle principali cause della morte della madre.

\subsection{Un figlio riconosciuto troppo presto}

Quando il padre di Olorin mise incinta Gnegnarella, egli era già promesso in sposa alla figlia di un Visconte non troppo distante, la quale tuttavia era ancora troppo giovane per il matrimonio. La storia con la giovane elfa, e ancor di più la decisione di riconoscere il figlio furono scelte avventate, legate all'impulsività del giovane, pericolose nella sua situazione.

Questo gesto portò ad una rottura del fidanzamento e attirò dicerie e giudizi negativi da parte della comunità nobiliare nonchè l'ira del padre. Ciò costrinse Karl a sposare una baronessa, di grado nobiliare quindi inferiore.

La baronessa in questione era Esther Galehorn, più vecchia di Karl di cinque anni e decisa fino in fondo a elevare il proprio status grazie a questa opportunità di matrimonio. Avuti due gemelli da Karl, un maschio ed una femmina, iniziò a lavorare per allontanare Olorin dalla corte, sfruttando anche l'astio del ragazzo nei confronti del padre.
