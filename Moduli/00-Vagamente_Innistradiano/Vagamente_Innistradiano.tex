\documentclass[10pt,twoside, twocolumn, openany]{dndbook}

\usepackage[english]{babel}
\usepackage[utf8]{inputenc}
\usepackage{lipsum}
\usepackage{listings}
\usepackage{flushend}
\usepackage{dblfloatfix}

\lstset{%
  basicstyle=\ttfamily,
  language=[LaTeX]{TeX},
}


\begin{document}

\section{Backbone: bozza}

In una cittadina da un paio di migliaia di abitanti c'è un santuario relativamente grande per via di una reliquia. Tale reliquia consiste di una spada ancora impugnata da una mano che non subisce in alcun modo l'influenza del tempo (DC 18 \textbf{Percezione(WIS)}: l'oggetto è strano, come un'illusione, ma non \textit{sembra} un'illusione. Se chi passa il tiro è competente in \textbf{Arcana(INT)}: non è completamente nel piano materiale).

La reliquia è inamovibile, e per tanto il santuario le è stato costruito attorno. Tale santuario è gestito da un gruppo di monaci: la chiesa sovrastante la cripta dove è contenuta la reliquia è l'unica sezione normalmente accessibile al pubblico. Dietro a tale tempio vi sono gli alloggi dei monaci ed il resto del monastero.

Inizialmente il party si trova nella città. Oltre dieci persone morte nell'ultimo mese spingono la guarnigione cittadina a bloccare gli accessi alla città. Inoltre diverse delle vittime vengono identificate come frequentatori o lavoratori orbitanti attorono alla chiesa, ma non essendovi prove sufficentemente strutturate per metter naso all'interno di questa, viene richiesta la collaborazione del party come intermediario.

Gli omicidi sono tutti violenti, e completamente gratuiti: persone qualunque, cittadini modello, iniziavano improvvisamente a farfugliare, gridare, e aggredire chiunque fosse a portata di mano, con forza sovraumana, per poi morire dopo meno di un'ora. Come già indicato, molti degli aggressori frequentavano più della media la cattedrale.

L'autopsia a questi individui rivelerebbe una cicatrice, come una marchiatura a fuoco, a forma di occhio sul cervello. Nessuna conoscenza sarà sufficente a stabilirne l'esatta origine. \textbf{Detect Magic} potrà rivelare una leggera traccia di magia, come un residuo, o come di una magia troppo debole per provocare una simile perdita di lucidità.
Tali individui avranno inoltre spesso le ossa spezzate e i muscoli ipetrofici e pieni di lesioni, oltre a eventuali ferite già presenti.

Se il party non vuole indagare ma se la vuole filare, qualunque strada tenterano di prendere incontreranno individui pronti a "trasformarsi" per aggredirli. More importantly: sono liberi di farlo. Scopriranno più avanti che all'incirca metà della popolazione si è massacrata a vicenda e il santuario è andato distrutto in un incendio

\begin{paperbox}{Le vittime}
La prima vittima è un fabbro che aveva probabilmente lavorato alla riparazione di alcuni cancelli nelle zone non aperte al pubblico.

\end{paperbox}

\clearpage

\section{Backbone Halloween 2020}

il party inizia in una taverna di Lonna, sono appena tornati da un avventura.
la città al momento ha pochi avventurieri perchè molti sono stati reclutati come volontari per sopprimere una tana di orchi a nord-est 

la guardia cittadina manda un ragazzetto ad annunciare che si cercano avventurieri per pattuglie straordinarie delle strade.
il motivo è che alcune persone hanno iniziato ad impazzire e uccidere la propria famiglia e conoscienze.

il party potrà fare o meno questa cosa. La pattuglia sarà non distante dalla taverna/ostello dove risiede il party: La donna che gestisce la cucina, moglie del proprietario e devota al culto del dio della creazione sarà impazzita.
Si sentono urla, la donna cerca di aggredire il marito poco dopo la chiusura mentre riordinavano e pulivano la cucina.
la donna ha forza sovraumana ed è apparentemente sproporzionata. Non articola parole sensate, sbava. Dopo un po', comunque sia, si irrigidisce, inizia a graffiarsi il viso e muore. 

il party viene sentito dalla guardia come testimoni, vengono informati degli altri casi: un fabbro, un mugnaio, un anziano devoto.
Le loro autopsie rivelavano la stessa ipertrofia dei muscoli, ferite al viso, e cranio deformato. Apparentemente chi ha condotto l'autopsia ha trovato quelli che sembrano occhi sulla superficie del cervello.

Il timore delle guardie è che si tratti di una maledizione o di un parassita, e per mancanza di maghi di alto rango propongono di chiedere alla Crona che abita a sud della città, in delle grotte nella foresta.
Il party parte (lol) accompagnati da un ufficiale comunale, con un bel carretto con un cadavere dentro, rassicurati dal capo delle guardie che la Crona avrebbe apprezzato.
La Crona Ellie Bubblemud non tratta gratis, ma questa volta fa un'eccezione: quello che sta accadendo la turba. Lonna è una città di medie dimensioni, priva di maghi competenti: lei è tollerata li, e dove troverebbe altrettanti poveracci da spennare?
Si tratta di un affare di reciproco vantaggio.
Il cadavere puzza di Reame Remoto da un metro di distanza, e lei sa perchè
quei tonti della chiesa adorano un pezzo di Edain pensando sia un santo 
un entità aliena, di altri piani
ormai un mostro 
questo è lacerato, a cavallo tra questo reame ed il reame remoto: sarà sufficente trascinare qui quel mostro e rispedirlo a casa. Come?
La Crona ha fortunatamente qualcosa di molto adatto: una corda fatata tessuta nell'argendo puro, da legare attorno alla reliquia per trascinare qui il Coso.
Il notaio chiederà ora di trattare con la Crona in privato e a nome della città. Dopo circa un'ora uscirà visibilmente provato, e sarà riluttante a dire cosa è stato concesso in cambio.
(cioè meno controlli su casinò e bordelli: la crona vuole disperati che vengano da lei)

Ritornati in città il caos regna: decine di persone sono diventate violente all'improvviso (tutta gente di chiesa) e saranno spronati ad andare rapidamente al monastero mentre la guardia cittadina si occupa dei disordini
Nel monastero mini dungeon in cui si ammazzano un po' di preti e poi sala finale. Qui notevole il fatto che i monaci non si scannino tra di loro, ma "continuino" le loro mansioni se non interrotti. 
Tuttavia alcuni muoiono per le mutazioni, quindi ci sono un po' di cadaveri

Nella sala finale il capo del monastero: lui è molto più ganzo degli altri; quindi non è diventato completamente privo di senno, ma sicuramente non è un lume di sanità mentale.
Questo Avrà perso quasi completamente i capelli e avrà gli occhi che dano sull'esterno del cranio: è un warlock, e non da vantaggio se accerchiato (vede ovunque)
I monaci obbediscono a lui. lui vede il grande essere al di là dello squarcio, e crede sia il suo dio.
sconfittolo, si risolve la bega, l'edain arriva, CD14 su saggezza per non essere terrorizzati sul posto, fissa il party, prova a muovere le mani come per la prima volta da secoli e salta; lasciando il piano.

THE END

\clearpage

\section{dettagli 2020}

\subsection{Setting}
La città di Lonna è una città a prevalenza umana di medie dimensioni alla base della bassa catena delle Scelide, montagne antiche, simili all'appennino, che separano l'entroterra dei regni umani da quelli più periferici, confinanti con la foresta di Quaine.
Si tratta di una città di confine, un tempo centro minerario e ormai presidio militare di scarsa rilevanza. Sono infatti decine di anni che i regni umani di questa zona non si muovono guerra ed in particolre il confine, qui, è tranquillo.

Recentemente nel nordest dei gruppi di orchi sono scesi dalle montagne, probabilmente riuniti da un Capitano, e, stabilitisi nelle miniere abbandonate scendono regolarmente a sud per intercettare e razziare le carovane provenienti da est.
Nonostante le dimensioni non esagerate, per la sua natura di città di confine, si tratta di una meta di passaggio per molti avventurieri, al momento impiegati per lo più come mercenari nella soppressione delle bande di orchi.

La principale attrazione della città è una antica reliquia: una mano lacerata che ancora stringe una spada. Apparentemente inattaccabile dal tempo e inamovibile nello spazio divenne il centro di un santuario del dio della Creazione.
Il santuario attrae diversi pellegrini ma ospita circa una decina di monaci, dediti al lavoro e alle celebrazioni religiose.

A differenza del regno di Oster in questa zona il culto del dio della creazione non è l'unico e la sua chiesa è meno istituzionalizzata, occasionalmente istanziandosi in ordini non riconosciuti dalla chiesa di Andorbrig.

\begin{monsterbox}{Ellie Bubblemud}
  \begin{hangingpar}
    \textit{Medium fey, legal evil}
  \end{hangingpar}
  \dndline%
  \basics[%
  armorclass = 17,
  hitpoints  = \dice{11d8 + 33},
  speed      = {30 ft.},
  ]
  \dndline%
  \stats[
    STR = \stat{18},
    DEX = \stat{12},
    CON = \stat{16},
    INT = \stat{13},
    WIS = \stat{14},
    CHA = \stat{14},
  ]
  \dndline%
  \details[
    skills = {Arcana +3, Deception +4, Perception +4, Stealth +3},
    senses = {darkvision 60 ft., passive Perception 14},
    languages = {Common, Draconic, Sylvan},
    challenge = {3},
  ]
  \dndline%

  % Traits
  \begin{monsteraction}[Amphibious]
    the hag can breath air and water
  \end{monsteraction}

  \begin{monsteraction}[Innate Spellcasting]
    the hag's innate spellcasting ability is Charisma (spell save DC 12). She can innately cast the following spells, requiring no material components:

    At will: \textit{dancing lights, minor illusion, vicious mockery}
  \end{monsteraction}

  \begin{monsteraction}[Mimicry]
    The hag can mimic animal sounds and humanoid voices. A creature that hears the sounds can tell they are imitations with a successful DC14 Wisdom(insight) check.
  \end{monsteraction}

  \monstersection{Actions}

  \monstermelee[
    name=Claws,
    mod=+6,
    reach=5,
    targets=one target,
    dmg=\dice{2d8+4},
    dmgtype=slashing,
  ]

  \begin{monsteraction}[Illusory appearence]
    The hag covers herself and anything she is
wearing or carrying with a magical illusion that makes her look
like another creature of her general size and humanoid shape.
The illusion ends if the hag takes a bonus action to end it or
if she dies.
The changes wrought by this effect fail to hold up to physical
inspection. For example, the hag could appear to have smooth
skin, but someone touching her would feel her rough flesh.
Otherwise, a creature must take an action to visually inspect
the illusion and succeed on a DC 20 Intelligence (Investigation)
check to discern that the hag is disguised.
  \end{monsteraction}

  \begin{monsteraction}[Invisible passage]
    The hag magically turns invisible until she
    attacks or casts a spell, or until her concentration ends (as
    if concentrating on a spell). While invisible, she leaves no
    physical evidence of her passage, so she can be tracked only by
    magic. Any equipment she wears or carries is invisible with her.  
  \end{monsteraction}

\end{monsterbox}


\begin{monsterbox}{Monaco sfigghy}
  \begin{hangingpar}
    \textit{Medium humanoid, neutral}
  \end{hangingpar}
  \dndline%
  \basics[%
  armorclass = 12 (natural),
  hitpoints  = \dice{3d8 + 3},
  speed      = {30 ft.},
  ]
  \dndline%
  \stats[
    STR = \stat{16}, % This stat command will autocomplete the modifier for you
    DEX = \stat{8},
    CON = \stat{16},
    INT = \stat{7},
    WIS = \stat{11},
    CHA = \stat{8},
  ]
  \dndline%
  \details[% If you want to use commas in these sections, enclose the description in braces.
    savingthrows = {Con +5},
    %skills = {Acrobatics +0, Animal Handling +0, Arcana +0, Athletics +0, Deception +0, History +0, Insight +0, Intimidation +0, Investigation +0, Medicine +0, Nature +0, Perception +0, Performance +0, Persuasion +0, Religion +0, Sleight of Hand +0, Stealth +0, Survival +0},
    %damagevulnerabilities = {},
    %damageresistances = {},
    %damageimmunities = {},
    %conditionimmunities = {},
    %senses = {darkvision 60 ft., passive Perception 10},
    languages = {Common, Deep speech},
    challenge = {1/2},
  ]
  \dndline%

  % Traits
  \monstersection{Actions}
  \begin{monsteraction}[Screech from the Deep]
    Once a day the sfigghy monk can emit a terrible screech uttering uncomprehensible words from the Deep.
    Whoever hears this sound must make a DC14 Wisdom saving trow or be shaken by this sound and get disadvantage on the next saving throw, attack roll or skill roll.
  \end{monsteraction}

  %\monstermelee calls \monsterattack with the melee option
  \monstermelee[
    name=Claws,
    mod=+5,
    reach=5,
    targets=one target,
    dmg=\dice{1d4+5},
    dmgtype=slashing,
  ]
 
\end{monsterbox}

\begin{monsterbox}{Monacapo}
  \begin{hangingpar}
    \textit{Medium aberration, neutral evil}
  \end{hangingpar}
  \dndline%
  \basics[%
  armorclass = 11 (natural) (13 with \emph{mage armor}),
  hitpoints  = \dice{3d8 + 9},
  speed      = {30 ft.},
  ]
  \dndline%
  \stats[
    STR = \stat{14}, % This stat command will autocomplete the modifier for you
    DEX = \stat{11},
    CON = \stat{16},
    INT = \stat{10},
    WIS = \stat{15},
    CHA = \stat{14},
  ]
  \dndline%
  \details[% If you want to use commas in these sections, enclose the description in braces.
    %savingthrows = {Str +0, Dex +0, Con +0, Int +0, Wis +0, Cha +0},
    %skills = {Acrobatics +0, Animal Handling +0, Arcana +0, Athletics +0, Deception +0, History +0, Insight +0, Intimidation +0, Investigation +0, Medicine +0, Nature +0, Perception +0, Performance +0, Persuasion +0, Religion +0, Sleight of Hand +0, Stealth +0, Survival +0},
    %damagevulnerabilities = {},
    %damageresistances = {},
    %damageimmunities = {},
    %conditionimmunities = {},
    senses = {darkvision 60 ft., passive Perception 12},
    languages = {Common, Deep Speech},
    challenge = {2},
  ]
  \dndline%

  \begin{monsteraction}[Telepathy]
    The Monacapo can speak telepathically with each creature that knows at least a language in a 30ft radious
  \end{monsteraction}

  \begin{monsteraction}[Devil's sight]
    The Monacapo can see in darkness, both magical and not magical, with a radious of 120ft.
   \end{monsteraction}

  % Traits
  \monstersection{Actions}
  \begin{monsteraction}[Pact weapon]
    The Monacapo can summon a magical Longsword as an action, that takes the apearence of the holy relic.
  \end{monsteraction}

  \begin{monsteraction}[Edgy Armor]
    The Monacapo can cast Mage Armor at will without consuming a spell slot and without the need for material components.
  \end{monsteraction}

  \begin{monsteraction}[Spellcasting]
    The Monacapo knows the following spells and cantrips and has 2 slots of second level to cast them:
    
    \textit{cantrips}: Eldritch Blast, Frostbite
    
    \textit{level 1}: Armor of Agathis, Dissonant Whispers, Tasha's Hideous Laughter

    Its spellcasting ability for it is Charisma (+4 to hit, DC=12)
  \end{monsteraction}

  %\monstermelee calls \monsterattack with the melee option
  \monstermelee[
    name=Summoned Longsword,
    mod=+4,
    reach=5,
    targets=one target,
    dmg=\dice{1d8+2},
    dmgtype=slashing,
    ordmg=\dice{1d10+2},
    ordmgwhen=if used with two hands
  ]

\end{monsterbox}



\end{document}
