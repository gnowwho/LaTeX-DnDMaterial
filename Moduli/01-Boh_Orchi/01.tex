\documentclass[10pt,twoside, twocolumn, openany]{dndbook}

\usepackage[english]{babel}
\usepackage[utf8]{inputenc}
\usepackage{lipsum}
\usepackage{listings}
\usepackage{flushend}
\usepackage{dblfloatfix}

\lstset{%
  basicstyle=\ttfamily,
  language=[LaTeX]{TeX},
}


\begin{document}

\section{Backbone: bozza}
Si era a Lonna nel monastero.
Il capo delle guardie deve dare 70Gp a testa al party. Il figlio dei locandieri è andato ad aiutare nella soppressione degli orchi.

Tzuur è stato ucciso: si trattava di un paladino degli Orchi che li aveva unificati e giudati verso sud. Ora si parla già di smontare i campi e lasciare agli avventurieri mercenari e i gruppi di guardia locale il resto delle soppressioni.
La popolazione locale, compreso il Barone di Lonna, Errol Tysel (che lo verrà a sapere solo tra un giorno o due) sono contrari al ritiro della guarnigione, ma il Marchese Grunbaum propende per il ritiro nella convinzione che gli orchi si ritireranno alla minima resistenza senza un leader.
In ciò non si sbaglia, ma è comunque vero che il danno prodotto dalle scorribande incontrollate durante la ritirata degli orchi è difficile da ignorare.

Il party dove sta? boh. Potrebbe venir raggiunto da un amico di famiglia dei proprietari del Guforso che li informano del fatto che Charlie sta partecipando alla campagna di soppressione e chiede loro di andare a prenderlo, per informarlo della morte dei genitori e riportarlo a casa.
Che ha questo tizio da offrire? Dinero? Una Canzone Ska? Magari un anello molto prezioso che rivela il pericolo che ha ottenuto per passamano, o perchè era un avventuriero finchè non si è preso una freccia nel ginocchio.



\begin{DndSidebar}{Charlie}
Figlio di Teresa e Owen, i locandieri del Guforso, è un giovane onesto e impulsivo che ha deciso di partecipare come volontario nella soppressione degli orchi ad est. 
\end{DndSidebar}

\begin{DndSidebar}{Robert}
    Il "tipo dell'amministrazione" si chiama Robert. Si tratta di un ometto di mezza età, praticamente calvo, acido e pragmatico. Non ha particolari obiettivi nella vita nè nel bene nè nel male, ma tiene alla propria tranquillità, che è disposto a difendere, con ogni mezzo, entro i limiti della legge.
\end{DndSidebar}



\clearpage

\section{Elaborazione migliore}



\clearpage

\section{dettagli 2020}

\subsection{Setting}
La città di Lonna è una città a prevalenza umana di medie dimensioni alla base della bassa catena delle Scelide, montagne antiche, simili all'appennino, che separano l'entroterra dei regni umani da quelli più periferici, confinanti con la foresta di Quaine.
Si tratta di una città di confine, un tempo centro minerario e ormai presidio militare di scarsa rilevanza. Sono infatti decine di anni che i regni umani di questa zona non si muovono guerra ed in particolre il confine, qui, è tranquillo.

Recentemente nel nordest dei gruppi di orchi sono scesi dalle montagne, probabilmente riuniti da un Capitano, e, stabilitisi nelle miniere abbandonate scendono regolarmente a sud per intercettare e razziare le carovane provenienti da est.
Nonostante le dimensioni non esagerate, per la sua natura di città di confine, si tratta di una meta di passaggio per molti avventurieri, al momento impiegati per lo più come mercenari nella soppressione delle bande di orchi.

La principale attrazione della città è una antica reliquia: una mano lacerata che ancora stringe una spada. Apparentemente inattaccabile dal tempo e inamovibile nello spazio divenne il centro di un santuario del dio della Creazione.
Il santuario attrae diversi pellegrini ma ospita circa una decina di monaci, dediti al lavoro e alle celebrazioni religiose.

A differenza del regno di Oster in questa zona il culto del dio della creazione non è l'unico e la sua chiesa è meno istituzionalizzata, occasionalmente istanziandosi in ordini non riconosciuti dalla chiesa di Andorbrig.

\begin{DndMonster}{Ellie Bubblemud}
  \begin{hangingpar}
    \textit{Medium fey, legal evil}
  \end{hangingpar}
 % \DndMonsterLine%
  \DndMonsterBasics[%
  armorclass = 17,
  hitpoints  = \DndDice{11d8 + 33},
  speed      = {30 ft.},
  ]
 % \DndMonsterLine%
  \DndMonsterAbilityScores[
      str = 18, 
      dex = 12,
      con = 16,
      int = 13,
      wis = 14,
      cha = 14,
  ]
  %\DndMonsterLine%
  \DndMonsterDetails[
    skills = {Arcana +3, Deception +4, Perception +4, Stealth +3},
    senses = {darkvision 60 ft., passive Perception 14},
    languages = {Common, Draconic, Sylvan},
    challenge = {3},
  ]
  %\DndMonsterLine%

  % Traits
  \begin{DndMonsterAction}{Amphibious}
    the hag can breath air and water
  \end{DndMonsterAction}

  \begin{DndMonsterAction}{Innate Spellcasting}
    the hag's innate spellcasting ability is Charisma (spell save DC 12). She can innately cast the following spells, requiring no material components:
    \begin{DndMonsterSpells}
        \DndMonsterSpellLevel{dancing lights, minor illusion, vicious mockery}
    \end{DndMonsterSpells}
  \end{DndMonsterAction}

  \begin{DndMonsterAction}{Mimicry}
    The hag can mimic animal sounds and humanoid voices. A creature that hears the sounds can tell they are imitations with a successful DC14 Wisdom(insight) check.
  \end{DndMonsterAction}

  \DndMonsterSection{Actions}

  \DndMonsterMelee[
    name=Claws,
    mod=+6,
    reach=5,
    targets=one target,
    dmg=\DndDice{2d8+4},
    dmg-type = Slashing,
  ]

  \begin{DndMonsterAction}{Illusory appearence}
    The hag covers herself and anything she is
wearing or carrying with a magical illusion that makes her look
like another creature of her general size and humanoid shape.
The illusion ends if the hag takes a bonus action to end it or
if she dies.
The changes wrought by this effect fail to hold up to physical
inspection. For example, the hag could appear to have smooth
skin, but someone touching her would feel her rough flesh.
Otherwise, a creature must take an action to visually inspect
the illusion and succeed on a DC 20 Intelligence (Investigation)
check to discern that the hag is disguised.
  \end{DndMonsterAction}

  \begin{DndMonsterAction}{Invisible passage}
    The hag magically turns invisible until she
    attacks or casts a spell, or until her concentration ends (as
    if concentrating on a spell). While invisible, she leaves no
    physical evidence of her passage, so she can be tracked only by
    magic. Any equipment she wears or carries is invisible with her.  
  \end{DndMonsterAction}


\end{DndMonster}




\end{document}