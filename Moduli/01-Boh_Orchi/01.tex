\documentclass[10pt,twoside, twocolumn, openany]{dndbook}

\usepackage[english]{babel}
\usepackage[utf8]{inputenc}
\usepackage{lipsum}
\usepackage{listings}
\usepackage{flushend}
\usepackage{dblfloatfix}
\usepackage[normalem]{ulem}

\lstset{%
  basicstyle=\ttfamily,
  language=[LaTeX]{TeX},
}


\begin{document}

\section{Backbone: bozza}
Si era a \textbf{Lonna} nel monastero.
Il capo delle guardie deve dare 70Gp a testa al party. Non lo farà se non avrà ricevuto la fune d'argento. Il figlio dei locandieri, \textbf{Charlie} è andato ad aiutare nella soppressione degli orchi.

\textbf{Tzuur} è stato ucciso: si trattava di un paladino degli Orchi che li aveva unificati e giudati verso sud. Ora si parla già di smontare i campi e lasciare agli avventurieri mercenari e i gruppi di guardia locale il resto delle soppressioni.
La popolazione locale, compreso il \textbf{Barone di Lonna, Errol Tysel} (che lo verrà a sapere solo tra un giorno o due) sono contrari al ritiro della guarnigione, ma il \textbf{Marchese Grunbaum} propende per il ritiro nella convinzione che gli orchi si ritireranno alla minima resistenza senza un leader.
In ciò non si sbaglia, ma è comunque vero che il danno prodotto dalle scorribande incontrollate durante la ritirata degli orchi è difficile da ignorare.

Il party dove sta? boh. Potrebbe venir raggiunto da un amico di famiglia dei proprietari del Guforso che li informano del fatto che \textbf{Charlie} sta partecipando alla campagna di soppressione e chiede loro di andare a prenderlo, per informarlo della morte dei genitori e riportarlo a casa.
Che ha questo tizio da offrire? Dinero? Una Canzone Ska? Magari un anello molto prezioso che rivela il pericolo che ha ottenuto per passamano, o perchè era un avventuriero finchè non si è preso una freccia nel ginocchio.

Se il party decide di andare a recuperare Charlie, dovrà raggiungere il campo allestito dalle forze del Marchese che raccoglie molti degli avventurieri della zona. Charlie sarà magari stanziato come guardia in un villaggio relativamente fuori dalla zona calda, un po' più a sud-est del campo: \textbf{Ghille} 


\begin{DndSidebar}{Charlie}
Figlio di Teresa e Owen, i locandieri del Guforso, è un giovane onesto e impulsivo che ha deciso di partecipare come volontario nella soppressione degli orchi ad est. 
\end{DndSidebar}

essendo la testa \sout{di cazzo} calda che è Charlie si ficcherà nei casini peggiori. Se il party decide di raggiungerlo lui accoglierà tristemente la notizia sostenendo di volerci dormire su; sarà già quasi sera. Il villaggio verrà attaccato dalla banda di Grok proprio perchè un obiettivo a basso rischio con solo un paio di guardie e pochi uomini armati. 
L'attacco sarà completamente diverso dalle usuali modalità degli orchi: avverrà in piccoli numeri, di notte e furtivamente.

\begin{DndSidebar}{Robert}
    Il "tipo dell'amministrazione" si chiama Robert. Si tratta di un ometto di mezza età, praticamente calvo, acido e pragmatico. Non ha particolari obiettivi nella vita nè nel bene nè nel male, ma tiene alla propria tranquillità, che è disposto a difendere, con ogni mezzo, entro i limiti della legge.
\end{DndSidebar}

Con questo attacco Grok spera di riuscire a uccidere ogni abitante, e bruciare il villaggio con soli tre orchi, dimostrando la validità dei suoi metodi e il suo valore come leader. 
Non si tratta di una mossa prudente, ma è un tentativo disperato per ottenere una posizione che potesse permettergli di rientrare nella "stanza dei bottoni" e recuperare, almeno in parte l'unità tra le tribu orcheste.

L'attacco non sarà la serie di omicidi silenti che Grok aveva sperato a causa di \textbf{Jura}, un orco fedele a \textbf{Vrisk} e abituato a metodi tradizionali, primo dei quali è "cos'è un'ordine?"
L'attacco causerà la morte di una guardia che era di turno e qualche civile, e si fermerà solo quando, grazie all'intervento del party Grok capirà di aver perso ogni vantaggio nell'attacco. Jura obietterà alla ritirata e potrebbe caricare. 

ovviamente quel coglione di Charlie seguirà nella notte Grok.


\clearpage

\section{Personaggi: Bozza}

\subsection{Tzuur e Grok}
L'orda di Orchi scesa ai piedi delle montagne per occupare le vecchie miniere di ferro era guidata da \textbf{Tzuur}, un possente campione orchesco che si era guadagnato il rispetto dei capiclan come ogni orco che si rispetti: a cazzotti.
In realtà questa impresa fu possibile solo grazie alle sottili (per gli standard orcheschi) macchinazioni di \textbf{Grok}, un orco ben più esile ma anche più astuto e crudele. 
Grok ha vissuto una vita da opportunista, sfruttando Tzuur per ottenere potere e protezione. Tzuur non era sicuramente il più sveglio degli orchi, ma non era ignaro di questa cosa, semplicemente riteneva che i consigli di Grok fossero preziosi e aveva imparato a suo modo a rispettarlo.
Dopo la morte di Tzuur la posizione di Grok divenne instabile. Cercò di mantenere uniti i clan, trovando un altro leader forte nel capotribù \textbf{Vrisk}, ma non riuscì nè a guadagnare la sua fiducia nè a dominare i tumulti delle tribù.
Se alcuni clan tornarono immediatamente alle montagne, molti si diedero al saccheggio scoordinato delle zone vicine, causando ulteriori tumulti e subendo gravi perdite.

Nonostante Vrisk non si fidasse di Grok, e generalmente non lo ascoltasse, lo accolse comunque tra le proprie fila, accettando la sua banda armata grazie al rispetto che ancora aveva per Tzuur. 
Questo permise a Grok di agire in modo relativamente indipendente e con quasi esclusivamente orchi di sua fiducia: non è infatti uso tra gli orchi coordinare in modo particolarmente fiscale le singole bande armate, che hanno per lo più un rapporto di "kinship" e fiducia reciproca di tipo feudale, ma una discreta indipendenza.

Grok tenterà di recuperare il proprio standing con attacchi audaci, possibili solo con metodi diversi da quelli comunemente impiegati dalla sua specie. Il reale obiettivo non è il potere fine a se stesso, ma l'unità delle tribù orchesche.
Grok è infatti convinto che l'unico modo in cui la sua comunità può emergere dalle sterili e dure montagne della Scelide è unirsi e diventare una forza che gli umani non possano considerare solo un fastidio. 
Sfortunatamente per lui questo sogno è un'illusione: le forze raccolte dagli orchi non potrebbero mai competere militarmente con uno, o peggio, un'alleanza, dei regni della zona. Ma Grok è solo sveglio, non esperto in politica estera.

\begin{DndSidebar}{La Banda di Grok}
 La banda armata di Grok ha una composizione relativamente inusuale per una squadra di Orchi: Un indubbio pregio di quest'orco è infatti trovare punti di forza in personaggi che hanno punti di forza diversi da quelli convenzionalmente rispettati degli orchi. \\
 \textbf{Grok:} Basically ladro \\
 \textbf{Hurr:} Basically ranger \\
 \textbf{Rashka:} Basically grosso guerriero cazzutissimo, ma femmina, e quindi ritenuta non adatta ai raid.  \\
 \textbf{Jura:} Orco normale, fedele a Vrisk, non in ottimi rapporti con Rashka e che non rispetta Grok
\end{DndSidebar}


\clearpage

\section{Elaborazione migliore}



\clearpage

\section{dettagli 2020}

\subsection{Setting}
La città di \textbf{Lonna} è una città a prevalenza umana di medie dimensioni alla base della bassa catena delle \textbf{Scelide}, montagne antiche, simili all'appennino, che separano l'entroterra dei regni umani da quelli più periferici, confinanti con la foresta di \textbf{Quaine}.
Si tratta di una città di confine, un tempo centro minerario e ormai presidio militare di scarsa rilevanza. Sono infatti decine di anni che i regni umani di questa zona non si muovono guerra ed in particolre il confine, qui, è tranquillo.

Recentemente nel nordest dei gruppi di orchi sono scesi dalle montagne, probabilmente riuniti da un Capitano, e, stabilitisi nelle miniere abbandonate scendono regolarmente a sud per intercettare e razziare le carovane provenienti da est.
Nonostante le dimensioni non esagerate, per la sua natura di città di confine, si tratta di una meta di passaggio per molti avventurieri, al momento impiegati per lo più come mercenari nella soppressione delle bande di orchi.

La principale attrazione della città è una antica reliquia: una mano lacerata che ancora stringe una spada. Apparentemente inattaccabile dal tempo e inamovibile nello spazio divenne il centro di un santuario del dio della Creazione.
Il santuario attrae diversi pellegrini ma ospita circa una decina di monaci, dediti al lavoro e alle celebrazioni religiose.

A differenza del regno di Oster in questa zona il culto del dio della creazione non è l'unico e la sua chiesa è meno istituzionalizzata, occasionalmente istanziandosi in ordini non riconosciuti dalla chiesa di Andorbrig.


\begin{DndMonster}{Ellie Bubblemud} %-----------------------------------------------------------
  \begin{hangingpar}
    \textit{Medium fey, legal evil}
  \end{hangingpar}
 % \DndMonsterLine%
  \DndMonsterBasics[%
  armorclass = 17,
  hitpoints  = \DndDice{11d8 + 33},
  speed      = {30 ft.},
  ]
 % \DndMonsterLine%
  \DndMonsterAbilityScores[
      str = 18, 
      dex = 12,
      con = 16,
      int = 13,
      wis = 14,
      cha = 14,
  ]
  %\DndMonsterLine%
  \DndMonsterDetails[
    skills = {Arcana +3, Deception +4, Perception +4, Stealth +3},
    senses = {darkvision 60 ft., passive Perception 14},
    languages = {Common, Draconic, Sylvan},
    challenge = {3},
  ]
  %\DndMonsterLine%

  % Traits
  \begin{DndMonsterAction}{Amphibious}
    the hag can breath air and water
  \end{DndMonsterAction}

  \begin{DndMonsterAction}{Innate Spellcasting}
    the hag's innate spellcasting ability is Charisma (spell save DC 12). She can innately cast the following spells, requiring no material components:
    \begin{DndMonsterSpells}
        \DndMonsterSpellLevel{dancing lights, minor illusion, vicious mockery}
    \end{DndMonsterSpells}
  \end{DndMonsterAction}

  \begin{DndMonsterAction}{Mimicry}
    The hag can mimic animal sounds and humanoid voices. A creature that hears the sounds can tell they are imitations with a successful DC14 Wisdom(insight) check.
  \end{DndMonsterAction}

  \DndMonsterSection{Actions}

  \DndMonsterMelee[
    name=Claws,
    mod=+6,
    reach=5,
    targets=one target,
    dmg=\DndDice{2d8+4},
    dmg-type = Slashing,
  ]

  \begin{DndMonsterAction}{Illusory appearence}
    The hag covers herself and anything she is
wearing or carrying with a magical illusion that makes her look
like another creature of her general size and humanoid shape.
The illusion ends if the hag takes a bonus action to end it or
if she dies.
The changes wrought by this effect fail to hold up to physical
inspection. For example, the hag could appear to have smooth
skin, but someone touching her would feel her rough flesh.
Otherwise, a creature must take an action to visually inspect
the illusion and succeed on a DC 20 Intelligence (Investigation)
check to discern that the hag is disguised.
  \end{DndMonsterAction}

  \begin{DndMonsterAction}{Invisible passage}
    The hag magically turns invisible until she
    attacks or casts a spell, or until her concentration ends (as
    if concentrating on a spell). While invisible, she leaves no
    physical evidence of her passage, so she can be tracked only by
    magic. Any equipment she wears or carries is invisible with her.  
  \end{DndMonsterAction}

\end{DndMonster}%-------------------------------------------------------------------



\begin{DndMonster}{Grok}%-----------------------------------------------------------
  \begin{hangingpar}
    \textit{Medium humanoid, Neutral evil}
  \end{hangingpar}
 % \DndMonsterLine%
  \DndMonsterBasics[%
  armorclass = 17,
  hitpoints  = \DndDice{11d8 + 33},
  speed      = {30 ft.},
  ]
 % \DndMonsterLine%
  \DndMonsterAbilityScores[
      str = 18, 
      dex = 12,
      con = 16,
      int = 13,
      wis = 14,
      cha = 14,
  ]
  %\DndMonsterLine%
  \DndMonsterDetails[
    skills = {Arcana +3, Deception +4, Perception +4, Stealth +3},
    senses = {darkvision 60 ft., passive Perception 14},
    languages = {Common, Orchish},
    challenge = {3},
  ]
  %\DndMonsterLine%

  % Traits
  \begin{DndMonsterAction}{--}
    --
  \end{DndMonsterAction}

  \begin{DndMonsterAction}{--}
    --
  \end{DndMonsterAction}

  \DndMonsterSection{Actions}

  \DndMonsterMelee[
    name=Claws,
    mod=+6,
    reach=5,
    targets=one target,
    dmg=\DndDice{2d8+4},
    dmg-type = Slashing,
  ]


\end{DndMonster}




\end{document}