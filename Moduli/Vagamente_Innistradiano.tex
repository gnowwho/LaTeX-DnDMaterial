\documentclass[10pt,twoside, twocolumn, openany]{dndbook}

\usepackage[english, italian]{babel}
\usepackage[utf8]{inputenc}
\usepackage{lipsum}
\usepackage{listings}
\usepackage{flushend}
\usepackage{dblfloatfix}

\lstset{%
  basicstyle=\ttfamily,
  language=[LaTeX]{TeX},
}


\begin{document}

\section{Backbone}

In una cittadina da un paio di migliaia di abitanti c'è un santuario relativamente grande, per via di una reliquia. Tale reliquia consiste di una spada ancora impugnata da una mano che non subisce in alcun modo l'influenza del tempo (DC 18 \textbf{percezione(WIS)}: l'oggetto è strano, come un'illusione, ma non \textit{sembra} un'illusione. Se chi passa il tiro è competente in \textbf{Arcana(INT)}: non è completamente nel piano materiale).

La reliquia è inamovibile, e per tanto il santuario le è stato costruito attorno. Tale santuario è gestito da un gruppo di monaci: la chiesa sovrastante la cripta dove è contenuta la reliquia è l'unica sezione normalmente accessibile al pubblico. Dietro a tale tempio vi sono gli alloggi dei monaci ed il resto del monastero.

Mentre il party staziona nella città diversi strani omicidi iniziano ad avvenire. Oltre dieci persone morte nell'ultimo mese spingono la guarnigione cittadina a bloccare gli accessi alla città e a chiedere collaborazione agli avventurieri locali, in attesa di una risposta dalla capitale, prevista comunque per almeno un'altro mese.

Gli omicidi sono tutti violenti, apparentemente scollegati tra loro, e completamente gratuiti: persone qualunque, cittadini modello, iniziavano improvvisamente a farfugliare, gridare, e aggredire chiunque fosse a portata di mano, con forza sovraumana, per poi morire dopo meno di un'ora.

L'autopsia a questi individui rivelerebbe una cicatrice, come una marchiatura a fuoco, a forma di occhio sul cervello. Nessuna conoscenza sarà sufficente a stabilirne l'esatta origine. Detect Magic potrà rivelare una leggera traccia di magia, come un residuo, o come di una magia troppo debole per provocare una simile perdita di lucidità.
Tali individui avranno inoltre spesso le ossa spezzate e i muscoli ipetrofici e pieni di lesioni, oltre a eventuali ferite già presenti.

Se il party non vuole indagare ma se la vuole filare, qualunque strada tenterano di prendere incontreranno individui pronti a "trasformarsi" per aggredirli. More importantly: sono liberi di farlo. Scopriranno più avanti che all'incirca metà della popolazione si è massacrata a vicenda e il santuario è andato distrutto in un incendio

Se il party inizia ad indagare arriverà a parlare con il signore locale che li reindirizzerà all'abate del santuario, suggerendo che almeno un paio di essi avessero contatti costanti con il monastero, e almeno un terzo, un fabbro, aveva probabilmente lavorato alla riparazione di alcuni cancelli nelle zone non aperte al pubblico. Detto ciò, tutta la popolazione cittadina frequenta più o meno spesso il templio, pertanto la pista è tutto sommato debole.
In realtà il signore della città vorrebbe semplicemente evitare di interrogare personalità vicine alla chiesa per evitare dissapori, e tenta pertanto di ottenere informazioni tramite 


\end{document}
